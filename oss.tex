\documentclass[10pt,a4paper]{article}
\usepackage[utf8]{inputenc}
\usepackage[german]{babel}
\usepackage[margin=0.4in]{geometry}
\usepackage{multicol}
\usepackage{xcolor}
\usepackage{enumitem}
\usepackage{titlesec}

% Balanced spacing
\setlength{\parindent}{0pt}
\setlength{\parskip}{3pt}
\setlist{nosep,leftmargin=14pt,itemsep=1pt}

% Colors
\definecolor{blue}{RGB}{0,51,102}

% Section formatting
\titleformat{\section}{\color{blue}\normalsize\bfseries}{\thesection}{0.5em}{}
\titlespacing*{\section}{0pt}{4pt}{2pt}

\begin{document}

\begin{center}
\textbf{\Large Betriebssysteme Cheat Sheet}\\
\small FernUni Hagen - Mündliche Prüfung
\end{center}
\hrule
\vspace{3mm}

\begin{multicols}{2}
\small

\section{Grundlagen}
\textbf{Betriebssystem:} Software zwischen Hardware und Anwendungen

\textbf{Hauptaufgaben:}
\begin{itemize}
\item Abstrakte Maschine bereitstellen
\item Ressourcenverwaltung (CPU, RAM, I/O)
\item Schutz zwischen Prozessen
\item Mehrprogrammbetrieb koordinieren
\end{itemize}

\textbf{Hardware-Konzepte:}
\begin{itemize}
\item Interrupts: Hardware → CPU → Handler
\item Dual Mode: Kernel- vs. User-Modus
\item Memory Protection: Adressbereichs-Schutz
\end{itemize}

\section{Prozesse \& Scheduling}
\textbf{Prozess} = laufendes Programm mit eigenem Speicher

\textbf{Zustände:} erzeugt → bereit → laufend → blockiert → beendet

\textbf{Scheduling-Algorithmen:}
\begin{itemize}
\item \textbf{FCFS:} Fair, aber Convoy-Effekt
\item \textbf{SJF:} Optimal, unbekannte Laufzeiten
\item \textbf{Round Robin:} Zeitscheiben, fair
\item \textbf{Priority:} Wichtigkeit + Aging
\end{itemize}

\textbf{Threads:} Teilen Code/Daten, eigener Stack/Register

\section{Speicherverwaltung}
\textbf{Paging:} Logische Seiten → Physische Frames
\begin{itemize}
\item Seitentabelle für Adressumsetzung
\item TLB als Hardware-Cache
\end{itemize}

\textbf{Virtueller Speicher:}
\begin{itemize}
\item Demand Paging: Laden bei Bedarf
\item Page Fault: Zugriff auf ausgelagerte Seite
\item Replacement: LRU (optimal), FIFO, Clock
\item Thrashing: Permanente Page Faults
\end{itemize}

\textbf{Formel:} Phys. Adresse = Frame × PageSize + Offset

\section{Synchronisation}
\textbf{Race Condition:} Ergebnis von Ausführungsreihenfolge abhängig

\textbf{Semaphore:}
\begin{itemize}
\item down(s): s--, blockiere wenn s $<$ 0
\item up(s): s++, wecke wartenden Prozess
\end{itemize}

\textbf{Klassische Probleme:}
\begin{itemize}
\item Producer-Consumer: Bounded Buffer
\item Dining Philosophers: Deadlock-Vermeidung
\item Readers-Writers: Mehrere Leser, ein Schreiber
\end{itemize}

\textbf{Deadlock - 4 Bedingungen:}
\begin{itemize}
\item Mutual Exclusion
\item Hold and Wait
\item No Preemption
\item Circular Wait
\end{itemize}

\textbf{Lösungen:} Prevention, Avoidance (Banker), Detection

\section{Dateisysteme}
\textbf{Festplatte:} Zugriffszeit = Seek + Rotation + Transfer

\textbf{Scheduling:} FCFS, SSTF, SCAN (Elevator)

\textbf{Datei-Allokation:}
\begin{itemize}
\item FAT: Zentrale Tabelle, linked list
\item i-nodes: Index-Blöcke, mehrstufig
\item NTFS: Master File Table
\end{itemize}

\textbf{I/O-Techniken:} Polling → Interrupts → DMA

\section{Sicherheit}
\textbf{Authentifizierung - 3 Faktoren:}
\begin{itemize}
\item Wissen (Passwort)
\item Besitz (Karte)
\item Eigenschaften (Biometrie)
\end{itemize}

\textbf{Access Control:}
\begin{itemize}
\item DAC: Discretionary (Owner decides)
\item MAC: Mandatory (System enforced)
\end{itemize}

\textbf{UNIX:} rwx für User/Group/Other

\section{Shell \& Kommandos}
\textbf{Prozess-Erzeugung:}
\begin{itemize}
\item fork(): Prozess duplizieren
\item exec(): Programm laden
\item wait(): Auf Child-Prozess warten
\end{itemize}

\textbf{I/O-Redirection:}
\begin{itemize}
\item 0=stdin, 1=stdout, 2=stderr
\item $<$ input.txt $>$ output.txt
\item Pipes: cmd1 | cmd2
\end{itemize}

\section{Prüfungstipps}
\textbf{Häufige Fragen:}
\begin{itemize}
\item Prozess vs. Programm erklären
\item Scheduling-Verfahren vergleichen
\item Semaphore bei Producer-Consumer
\item 4 Deadlock-Bedingungen nennen
\item Paging-Mechanismus erläutern
\item fork()/exec() Ablauf beschreiben
\end{itemize}

\textbf{Antwort-Schema:}
1. Definition 2. Beispiel 3. Vor-/Nachteile 4. Alternativen

\textbf{Wichtige Zahlen:}
\begin{itemize}
\item Page Size: 4 KB
\item Disk Access: ca. 10 ms
\item Context Switch: 1-10 mikrosec
\item Time Quantum: 10-100 ms
\end{itemize}

\end{multicols}

\end{document}